\chapter{La energía}


\section{¿Qué es la energía?}

La energía es todo aquello que puede producir movimiento o generar calor. Cuando la energía produce un movimiento, por ejemplo al pedalear una bici, decimos que se está produciendo un trabajo. Pero también empleamos la energía para encender la calefacción de nuestras casas.

\subsection{Tipos de energía}

Existen diferentes tipos de energía según la fuente y el efecto que produce.
\begin{itemize}
	\item{La energía calorífica se transmite desde los objetos calientes a los fríos. Es decir, el frío no existe, es la ausencia de calor, y los cuerpos calientes los que transmite el calor a los fríos.}
	\item{La energía luminos es la que emiten los objetos luminosos como el Sol, las bombillas o los LEDs.}
	\item{La energía eléctrica es la que proporciona corriente eléctrica, y lo transmitimos a través de cables eléctricos. Hoy en día la mayor parte de las máquinas, incluso algunos coches, funcionan con electricidad.}
	\item{La energía mecánica es la que posee un cuerpo que se mueve o puede ponerse en movimiento, como la energía del agua o del viento en movimiento.}
\end{itemize}


Cualquiera de estas formas de energía se puede transformar unas en otras. Por ejemplo, la energía eléctrica se puede transformar en energía calorífica cuando conectamos la estufa a la corriente eléctrica. Y la energía mecánica se puede transformar en energía eléctrica, por ejemplo cuando se mueve las aspas de un generador eólico (molino de viento).


\section{Las fuentes de energía}

Las fuentes de energía son los recursos naturales que podemos aprovechar para obtener energía y transformarla, principalmente en energía eléctrica, que es la más fácil de transportar de un sitio a otro.

Las fuentes de energía la podemos clasíficar en renovables y no renovables.

\subsection*{Fuentes de energía no renovables}

Las fuentes de energía no renovables son aquellas que se encuentran en cantidades limitadas o bien su proceso de fabricación se ha producidor durante varios millones de años, como ocurre con el carbón, el gas natural y el petróleo.

El carbón se emplea hoy día en la industria y para producir electricidad, aunque debido a su alto nivel para producir gases contaminantes cada vez se usa menos. El petróleo y sus derivados como gasolinas, diesel o queroseno, se emplea como combustibles para medios de transporte como coches, motos, aviones o maquinaria agrícola. 

Estos vehículos, han ido evolucionando con el tiempo para producir cada vez menos gases contaminantes y requerir un menor consumo.

\subsection*{Fuentes de energía renovables}

Las fuentes de energía renovables son las que se renuevan de forma contínua, por lo que son inagotables. Las fuentes de energía más importantes son el Sol, el viento y el aprovechamiento de las corrientes de agua ya sea en ríos o mares.

En las centrales hidroeléctricas se emplea la energía mecánica del agua en movimiento para generar energía eléctrica. 

La energía del sol puede recogerse mediante placas solares para calentar agua y la calefacción. Pero también se puede aprovechar mediante paneles fotovoltáicos para producir electricidad.

El viento, son masas de aire que se mueven, por lo que podemos aprovechar esta energía mecánica para usarla como energía eléctrica.



\begin{recuerda}{Recuerda}
Puesto que ha día de hoy seguimos usando muchas fuentes de energía no renovables, no debemos derrochar nada. Por eso el ahorro energético debe ser una norma fundamental del día a día, la mejor forma de ahorrar energía es usar el transporte público y apagando las luces y electrodomésticos cuando no se necesiten.
\end{recuerda}