\chapter{El calor}

\section{¿Qué es el calor?}

El calor es una forma de energía que en nuestra vida cotidiana usamos para calentarnos, cocinar e incluso obtener electricidad. El calor siempre se transmite desde los objetos calientes a los fríos, es decir que cuando tenemos frío, realmente lo que pasa es que estamos perdiendo nuestro calor corporal; en cambio, cuando tenemos calor, lo que ocurre es que el objeto caliente nos pasa calor a nosotros. Esto es lo que ocurrire en verano cuando nos ponemos a tomar el Sol.

La cantidad de calor que posee un cuerpo es lo que llamamos temperatura, y para medirla usamos un termómetro.

\section{¿Cómo se transmite el calor?}

Ya hemos visto que el calor pasa siempre del foco caliente al foco frío, pero cómo ocurre esto. Pues la trasmisión de calor puede ocurrir mediante tres mecanismos diferentes: \textbf{conducción}, \textbf{convección} y \textbf{radiación}.

\subsection{Conducción}

En los objetos sólidos, el calor se transmite por coducción. Esto es lo que ocurre cuando sujetas un objeto metálico por un extremo y pones el otro extremo al fuego. Al final, aunque la llama no toca el extremo por el que lo sujetas, acabarás quemándote, ya que el calor se ha ido transmitiendo. Pero este fenómeno no ocurre con todos los objetos, si en lugar de un alambre cogemos un palo de madera, al ponerlo al fuego el extremo por el que lo sujetas no se calentará. ¡Ojo, aunque no se caliente, si se puede quemar, porque es madera!

A los objetos como  el alambre que si transmiten el calor les llamamos \textbf{conductores}, todos los metales son conductores. Mientras que a los objetos que no transmite en calor los llamamos \textbf{aislantes}, la madera, el plástico o el poliestireno (el corcho blanco de embalar), son buenos aislantes del calor.

\subsection{Convección}

Los objetos en estado líquido o gaseosos, pueden transmitir el calor por convección. Por ejemplo, al calentar agua en una olla la parte de abajo se calienta y sube, mientras que la superficie con agua fría baja, producciendo una circulación de agua que llamamos corrientes de convección. Esto mismo también ocurre en la atmósfera, produciendo fenómenos como el viento.


\subsection{Radiación}

En cambio, cuando no existe un contacto directo con la fuente de calor, también se puede transmitir calor por radiación. De esta forma el Sol calienta nuestro planeta. Es el mecanismo más débil para transmitir calor, ya que al no existir contacto se pierde efectividad.



\section{¿Qué efectos produce el calor?}

Seguro que alguna vez has intentado abrir una puerta de acero en verano y se atasca, cosa que no ocurría con esta misma puerta en invierno. Esto se debe a que el calor al modificar la temperatura de los objetos, también es capaz de variar su tamaño. Es lo que llamamos \textbf{dilatación} y \textbf{contracción}.

\subsection{La dilatación}

La dilatación consiste en el aumento de tamaño de un objeto al aumentar su temperatura, es decir, al calentarse. Esto ocurre en los raíles de las vías del tren así como en los puentes, por esto cuando se monta un puente o una vía de tren, se deja un hueco entre pieza y pieza. 

\subsection{La contracción}

En el otro lado, la contracción es la disminución de tamaño de un objeto debido a la disminución de su temperatura, es decir, cuando se enfría. 



\section*{Actividades}

\begin{enumerate}
\item{Completa las siguientes frase:
\\
El \rule{4cm}{0.2mm} se transmite de los cuerpos calientes a los cuerpos fríos. 
La cantidad de \rule{4cm}{0.2mm} que tiene un cuerpo es la \rule{4cm}{0.2mm}.
La \rule{4cm}{0.2mm} tiene lugar cuando un cuerpo aumenta de tamaño, debido a un aumento de temperatura.
La \rule{4cm}{0.2mm} es el efecto contrario y se produce cuando el cuerpo se enfría.
}

\item{Dí que forma de transmisión de calor corresponde a cada dibujo.}

%Poner dibujos.

\item{Observa la image y explica por qué se deja estos espacios en los puentes.}




\end{enumerate}