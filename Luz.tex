\chapter{La luz y el sonido}

\section{¿Qué es la luz?}

La luz es una forma de energía que nos permite ver todo lo que nos rodea como las formas y los colores de los objetos. Se propaga a una velocidad muy rápidca, casi instantánea para nuestros ojos, en el aire viaja a una velocidad de casi 300 000 km/s, es decir, en un solo segundo ha recorrido 300 000 km, o lo que es lo mismo, en un solo segudno daría la vuelta al mundo siete veces y media.

\section{¿Qué objetos transmite luz?}

Los objetos que emiten luz propia les llamamos \textbf{focos luminosos}, y los podemos diferenciar en:

\begin{itemize}
\item \textbf{Foco luminoso natural:} son aquellos que emiten luz propia de forma espontánea como el Sol. 

\item \textbf{Foco luminoso artificial:} son aquellos que emiten luz en determinadas condiciones, como una bombilla cuando se le suministra electricidad.

\end{itemize}


\section{¿Cómo se propaga la luz?}

En el aire, que es donde normalmente se propaga la luz, lo hace siempre en todas las direcciones y en línea recta. A cada una de estas líneas rectas les llamamos rayo luminoso.


\section{¿Cómo se comportan los objetos ante la luz?}

Los cuerpos que no emiten luz solo se puede ver cuando los iluminamos con un foco de luz. Estos cuerpos lo que hacen es reflejar parte de la luz que reciben y absorber la otra parte. Ante este hecho, podemos clasificar los cuerpos en \textbf{opacos}, \textbf{translúcidos} y \textbf{transparentes}.

\begin{itemize}
\item \textbf{Los cuerpos opacos:} son aquellos que no dejan pasar la luz. Cuando los rayos de luz llegan a un cuerpo opaco, estos rebotan.

\item \textbf{Los cuerpos translúcidos:} son aquellos que dejan pasar algo de luz, por lo que las imágenes que se perciben detrás de ellos son imágenes borrosas. Esto ocurre con algunos líquidos y vidrios como los de las mamparas de baño.


\item \textbf{Los cuerpos transparentes:} son aquellos que permiten el paso de toda la luz, por lo que los objetos de detrás de ellos se perciben de forma nítida. Por ejemplo, el vidrio de las ventanas, el agua clara o el aire.
\end{itemize}